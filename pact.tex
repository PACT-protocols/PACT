\documentclass{article}
\usepackage[english]{babel}
\usepackage[utf8]{inputenc}
\usepackage{hyperref}
\usepackage{enumitem,kantlipsum}
\usepackage{graphicx, color}
\newcommand{\TODO}[1]{ {\color{blue} #1 }}


\title{{\Huge PACT\/}:   {\Huge P\/}rivacy Sensitive Protocols {\Huge A\/}nd Mechanisms
\\for Mobile {\Huge C\/}ontact {\Huge T\/}racing }
\date{}

\setlength{\oddsidemargin}{0.25 in}
\setlength{\evensidemargin}{-0.25 in}
\setlength{\topmargin}{-0.6 in}
\setlength{\textwidth}{6.5 in}
\setlength{\textheight}{8.5 in}
\setlength{\headsep}{0.75 in}
\setlength{\parindent}{0 in}
\setlength{\parskip}{0.1 in}


\begin{document}
\maketitle

\emph{This is a working document.}

As nations are seeking to circumvent devastating death tolls from COVID-19, many are resorting to \emph{mobile-based, contact tracing technologies} as a key tool in mitigating the pandemic. Harnessing mobile computing technologies is an obvious means to dramatically scale-up conventional epidemic response strategies to do tracking at population scale. However, straightforward and well-intentioned contact tracing applications can invade personal privacy and provide justification for data collection and mass surveillance that are inconsistent with the civil liberties that citizens will and should expect---and demand. To be effective, acceptable, and consistent with the need to observe commitments to  privacy, we must leverage designs and computing advances in privacy and security. In cases where it is valuable for individuals to share data with others, systems must provide voluntary mechanisms in accordance with ethical principles of personal decision making, including disclosure, and consent. We focus here on privacy-sensitive mobile tracing technologies that are designed to mitigate COVID-19, while upholding civil liberties. We refer to efforts to identify, study, and field  privacy-sensitive technologies, architectures, and protocols in support of mobile tracing as PACT (\emph{P}rivacy sensitive protocols \emph{A}nd standards for mobile \emph{C}ontact \emph{T}racing).

\begin{center}
\emph{The objective of PACT is to set forth transparent privacy and
  anonymity standards,\\
  which permit adoption of mobile contract tracing efforts while upholding civil liberties.}
\end{center}

We seek to specify a set of protocols and standards that achieve
these objectives. We also wish to provide a means to support the use over a population of multiple, mobile contact tracing applications (running on different phones) via providing interoperating capabilities that assure consistency with regards to the goals of privacy. 

Conventional contact tracing strategies executed by public health organizations operate as follows: Positively tested citizens are asked to reveal (voluntary, or enforced via public health policy or by law depending on region) their contact history to public health officers. The public health officers then inform other citizens who have been exposed to the virus based on co-location, via some definition of co-location, supported by look-up or inference about locations. The citizens deemed to have been exposed are then asked to take appropriate action (often to either seek tests or to quarantine
themselves and to be vigilant about symptoms).  It is important to emphasize that the current approach \emph{already} makes a tradeoff between the privacy of an positively tested individual and the benefits to society.

We describe a mobile contact-tracing application that seeks to augment the services provided by public health offices, by enabling the following capabilities via computing and communications technology:

\begin{itemize}
\item \textbf{Privacy-sensitive, mobile tracing:}  This protocol is to help automate what is already being done, while attempting to address similar privacy tradeoffs.
The app can enable someone who has become ill, or who has received confirmation of infection with a positive test for COVID-19, to voluntarily and anonymously share information that may be relevant to the wellness of other individuals.  In particular, a system can manage, in a privacy-sensitive manner, data about individuals who came in close proximity to them over a period of time (e.g., the last two weeks), even if there is no personal connection between these individuals.
%\footnote{Medical criteria for determining if an individual is at risk is "within 6 feet of a COVID-19 positive person for over 10 minutes".}. 
Crucially, as with conventional contract tracing done by public health services, the alert mechanism has a cryptographic implementation so as to ensure that positives cases remain anonymous to the general public and that the alerts to exposed citizens are informed in a secure manner. Individuals who share information do so with disclosure and consent around potential risks of private information being shared. We further discuss disclosure and security concerns in Section~\ref{sect:FAQ}. We focus on a proximity-detecting--centric procedure, based on Bluetooth (see Section~\ref{sect:Bluetooth}). Beyond this methodology, we note that solutions are also feasible based on the private and personal storage of absolute position sensing data (e.g., GPS, dead-reckoning, Wifi-signals, and their combination) on individuals' own devices in advance of voluntary sharing by people who test positive (see Section~\ref{sect:GPS}).


\item \textbf{Mobile-assisted contact tracing interviews:}  If a citizen becomes ill, he or she can use the app to improve the efficiency and completeness of manual contact tracing interviews. Privacy here is maximized by ensuring that all the data remains on the user's device, except for what they voluntarily decide to reveal to healthcare authorities in order to enable contact tracing. In advance of their making a decision to share, they are informed about how their data will be used and the potential risks of sharing.

\item \textbf{Narrowcast messages:}  Public health authorities can broadcast
  custom-tailored messages to specific, relevant subsets of citizens.  For example, the following message might be issued: ``If you visited
  the X Eldercare Center between March 7th and 10th, please email yy@hhhealth.org''.  Here, it is a citizens voluntary choice to search these messages for what is relevant to them or to set up automated alerting based on static (e.g., by geographic region) or dynamic (e.g., by sets of locations and times that are included in their personal dataset) filters to guide alerting and awareness.
  
\end{itemize}

Importantly, the smartphone app, by default, keeps \emph{all} personal data on a citizens' phones, while enabling these key capabilities; information is shared via voluntary disclosure actions taken, with the understandings relayed via careful disclosure. For example, if someone never tests positive for COVID-19 or tests positive but decides not to use the system, then *NO* data is ever sent from their phone to any remote servers. The data on the phone can be encrypted and can be set up to automatically time out based on end-user controlled policies.  This would prevent the dataset from being accessed or requested via legal subpoena or other governmental programs and policies.

\emph{From a civil liberties standpoint, the privacy guarantees these protocols ensure are consistent with the disclosures already extant in contract tracing methods done by public health services} (where some information from a positive tested citizen is revealed to other exposed citizens). In short, we seek to empower public health services, while maintaining civil liberties.
 
\section{FAQ: Privacy, Security, and Disclosures} \label{sect:FAQ}

Before we start this discussion, it is helpful to consider one consistent principle which the proposed protocols respect, which is also consistent with current contract tracing approaches,

\begin{center}
\emph{If you do not report as being positive, then no information of yours will leave your phone.}
\end{center}

In particular, with conventional contract tracing, it is only positively tested individuals who reveal information to the public health authorities. With the above, the discussion at hand largely focuses on what can be inferred when a positive disclosure occurs along with how a malicious user can impact the system.

We focus the discussion on the "mobile tracing" protocol for the following reasons:  "Narrowcasting" allows people to listen for events in their region, so it can viewed as a one way messaging system. For "mobile-assisted interviews," all the data remains on the user's device, except for what they voluntarily reveal to healthcare authorities in order to enable contact tracing.


%The mobile tracing protocol is the most important protocol to examine for implications to the basic principles of civil liberties.

\textbf{Privacy and Disclosure.} We start first with what private information is protected and what is shared voluntarily, following disclosure and consent. In the below, we refer to an exposed citizen as one who has been in contact with an individual who has tested as positive for COVID-19 (under criteria as defined by public health programs, e.g., "within 6 feet for over 10 minutes").

%\item Will my previously visited locations be available to the public?

%No. All your prior locations will be held securely on your phone. Furthermore, they are deleted from your phone after two weeks time time.

\begin{enumerate}[leftmargin=*]
\item If I am tested positive and I voluntarily disclose this information, what is revealed to others?

Any other citizen who uses the app (or an interoperable app) who has been exposed is notified. In some versions the time(s) that exposure occurred may be shared. Additional information that can be revealed, e.g., to public health organizations, includes aggregating statistics, such as the approximate number of people who have been infected. However, in the basic mobile tracing system that we envision, beyond risk to specific individuals, no information is revealed to any other citizens or entities (authorities, insurance companies, etc). 

\item If I am notified that I have been exposed, what is disclosed to the general public or to the authorities? 
 
The protocol does not transmit \emph{any} any of your private information to any public database or any other third party. This also true for unexposed citizens is as well. Recall that "If you do not report as being positive, then no information of yours will leave your phone."

\item For the automated tracing procedure, what data is stored on my phone? What data is stored publicly about me?

Each phone will broadcast a (pseudo) random signal (using Bluetooth) to other nearby phones. All phones will remember (from the last two weeks) the signals that they send and those signals they receive. These broadcast signals can be thought of as non-identifying random numbers. 
%\TODO{Y: It may make sense to emphasize that the signal each phone broadcasts changes every x minutes, so these signals are not only anonymizers, but cannot be linked to any user.}  
If someone test positives and chooses to disclose this to help others, they will then disclose \emph{only} their broadcasts in the last two weeks. Then other phones can check if they ``heard'' any of these broadcasts; if they did, then they can deduce that they have been in close proximity to a positively tested user.  Standard cryptographic procedures can prevent identity theft when the positive citizen uploads their broadcasts, which we now discuss.

\iffalse
\item Is any of my location information ever broadcast to the world?

No location information ever leaves your phone in the mobile-tracing protocol.  This one advantage of this particular Bluetooth beaconing based approach.   One could design GPS-centric approaches to mobile-tracing. Here, all individuals geo-location information would have to leave their phones; the approach would have to heavily rely on secure cryptography for privacy.
\fi

\end{enumerate}

\textbf{Security.} Now let us directly address what happens if malicious hackers, governments, or organizations compromise the system. In some cases, cryptographically secure procedures can prevent certain attacks, and, in other cases, malicious disclosure of information is preventable because the protocol stores very little data off your device. Only cryptographically secure data from positively confirmed individuals is ever stored off of any device.
%One interesting functionality is that the protocol can allow for both self-reporting and for medically confirmed reporting.  The former might be important where 

\begin{enumerate}[leftmargin=*]

\item If you are negative, can a malicious citizen listen to your phone's broadcasts and misreport you as being a positive?
%Can a hacker misreport negative individuals as medically confirmed positive? 

   No, this is not possible. These protocols are based on well-studied methods, which are also used for other security purposes including online banking. In fact, even if the malicious party records all Bluetooth signals going into and out of your phone, this is not possible.   To prevent this attack, your phone must securely keep one random number on its own device; with this random number, standard cryptographic mechanisms will prevent you from identity theft. 
   
   \iffalse 
   Again, recall that since you have not reported as being positive, then no information of yours will leave your phone\footnote{Technically, the Bluetooth protocol used is one where random signals are sent from your phone. However, these signals are random and reveal no information about any information on your phone, under standard  cryptographic assumptions.}.
   \fi
   
\iffalse
These protocols are based on well-studied methods, which are also used for other security purposes including online banking. In fact, even if the malicious party records all Bluetooth signals going into and out of your phone, this is not possible.  
To prevent this attack, your phone must securely keep one random number on its own device; with this random number, standard cryptographic mechanisms will prevent you from identity theft. 
\fi

\iffalse
\item Can a malicious individual or the government gain access to the public data, that is used to alert exposed citizens? 
%What can they learn?

Yes. In fact, they already have access to this data --- the security of the protocol is stronger due to the fact this data set is publicly available. The government/hacker learns a sequence of (encrypted) random identifiers that have been broadcast by positively reporting individuals. 
\fi

\iffalse
The government/hacker learns a sequence of random identifiers (and public health message) that have been broadcast, over time, in proximity of individuals who tested positive. The location of the individual who tested positive could be inferred if these random identifiers were collected separately at specific locations. The individual remains otherwise anonymous. The security of the system relies on the fact this data set can be made public.
\fi

\item Can a positive citizen's identity, who chooses to report being positive, be inferred by others? 

We note that de-anonymization is possible and is a risk to volunteers who would prefer to remain anonymous. An individual who had been in close proximity to a positive individual may be able to infer the identity of this positively tested individual. 
Furthermore, two (or more) users who have been in proximity of the same positive may be able to jointly collude to identify the positive.
However, the positive's identity will never be explicitly broadcast. In fact, identities are not even stored in the dataset; it is only the positive's random broadcasts that are stored.

It should be noted that preventing proximity based identification of this sort is simply not possible to avoid in any protocol, even in manual contract tracing as done by public health services. This is because of the simple fact that, if an individual has been alerted to being exposed,  then the alert itself may contain information that is correlated with identifying information.  Preventing such disclosures is not possible, even under cryptographic procedures. 

%This is because the simple fact that an individual has been alerted to being exposed contains information that itself could reveal identifying information.  Preventing such disclosures is not possible, even under cryptographic procedures.

\iffalse
\item Can a malicious citizen eavesdrop on the Bluetooth signals\footnote{The signals we mean here are those signals that broadcast under the mobile tracing protocol.} that my phone is broadcasting publicly (to ping other nearby phones) and then use this information in any way against me or any other citizens?

No, this is not possible. Your broadcasts look like random numbers (gibberish) to any other users who eavesdrop, so the malicious citizen will only hear random numbers. These numbers convey by themselves no information, and cryptographic mechanisms prevent them being used for other malicious goals. For example, to frame you to be a positive subject and trigger an exposure warning. 
\fi

\end{enumerate}

\textbf{Hospital Reporting and Self Confirmed Reporting.} 
Given that we would like the protocol to be of use to different states and countries, we seek an approach which allows for both security in reporting and for flexibility from the app designer in regions where it may make sense to also consider self confirmed reporting or self confirmed symptoms.

Does the app support both medically confirmed positive tests and self confirmed positives tests?
\begin{enumerate}[leftmargin=*]

\item Yes, it supports both. The uploaded files will contain signatures from the uploading party (i.e. from a hospital or from the application). This permits a protocol to use information from hospitals and information from individuals in possibly different manner. In less developed nations, it may be helpful to permit the application to allow for reports based on less reliable signatures. 

\end{enumerate}

\textbf{Surveillance.} 
Finally, it is helpful to contrast the goals in PACT to several high-profile tracing protocols and mechanisms employed to date in other countries.

\begin{enumerate}[leftmargin=*]
\item How does PACT contrast to approaches of other governments?

South Korea, Israel, and United Kingdom all have passed emergency laws which permit the state to use any telecommunication information, which gives the state significant surveillance powers. These laws set worrisome precedents for privacy and civil liberties. PACT is designed to provide valuable services for mitigating the flow of infection through a population, while taking a privacy-first approach to tracing, focusing on the critical and central challenge of privacy. We seek to provide transparent procedures and protocols so that the security and privacy concerns can be upheld for any citizen to examine and to engage with in a voluntary and informed manner.  In this spirit, the privacy guarantees of PACT are designed to be consistent with the disclosures already extant in contract tracing methods done by public health services. 

\end{enumerate}

\section{Protocols and Standards} 
We now overview these three functionalities.


\subsection{Privacy-sensitive mobile tracing} \label{sect:Bluetooth}

We describe a privacy-sensitive, voluntary, community-sourced approach to sharing information. 
%The system can be based on individually and privately collected location information, whether by GPS, Wifi locations or other %beacons.  In another example implementation, 
Proximity-detecting signals are collected via Bluetooth or ultrasonics.  If a user is both infected and is willing to warn others who may have been at risk via proximity to the user, then anonymized information is uploaded to a server to warn other users of potential exposure. 

%We note that in the following description we do not discuss the mechanism of {\em how} reporting is managed -- i.e., %self-reporting vs. health-care provider assisted -- as this is orthogonal to the specification of the protocol. However, we %envision both options being possible, and depending on the type of reports, a signature is appended to upload to the server (with %an app-specific signing key in case of self reporting, and a key from a health-care provider otherwise.

We now describe our proposed protocol. (Lower-level technical details are omitted, e.g., how values are broadcast. Further, it is assumed the communication between users and the server is protected cryptographically by other means, usually TLS. 


\newcommand{\id}{\mathrm{id}}
\newcommand{\dt}{\mathrm{dt}}
\noindent {\bf Protocol description.} We first describe a variant of the protocol {\em without} entry validation, and discuss how to enhance the server side to allow for this. Further, we assume a relevant infection window (typically, two weeks). In particular, we fix an understood time unit $\dt$, and define $\Delta$ such that $\Delta \cdot \dt$ equals the infection window.


\begin{itemize}
    \item {\bf Pseudorandom ID Generation.} Every user generates and periodically broadcasts a sequence of pseudorandom IDs $\id_1, \id_2, \ldots$. These IDs are fixed-length $n$-bit strings (typically, $n = 128$ bits). The $i$-th $\id_i$ is broadcast at any time in the window $[t_0 + \dt\cdot (i-1), t_0 + \dt\cdot i[$, where $t_0$ is the start time. Their generation relies on a length-doubling function $f: \{0,1\}^n \to \{0,1\}^{2n}$. The user initially generates a random seed $S_0$, and then, after $i$ time units, computes
\begin{displaymath}
(S_i, \id_i) \gets f(S_{i-1}) \;.
\end{displaymath}
After $i$ time units, the user only stores $S^* \gets S_{\max\{i-\Delta,0\}}$ and the current $S_i$. Time units are absolute time here - if the device was powered off or the application disabled, we need to advance to the appropriate $S_i$. The function $f$ is obtained from an appropriate hash function. For example, if $n = 128$, we can use $f(x) = \textrm{SHA-256}(x)$.\footnote{The key property is that each $\id_i$ is pseudorandom  as long as $S_0, S_1, \ldots, S_{i-1}$ are hidden, and even if any of the values $S_j$ for $j \geq i$ are revealed.}

%For efficiency reason, we note that the time %intervals to advance from $S_{i-1}$ to $S_i$ may be %longer than the intervals between Bluetooth %broadcasts, in which case the same $\id_i$ is broadcast repeatedly. 
\item {\bf Pseudorandom ID collection.} For every $\id$ broadcast by a device in its proximity at time $t$, a user stores a pair $(\id, t)$ in its local storage $\mathcal{S}$.
\item {\bf Reporting upon positive test.} To report a positive test, the user uploads $(S^*, t_{start}, t_{end})$ to the server, which appends it to a public list $\mathcal{L}$, where $t_{start}$ is time when $S^*$ was generated, and $t_{end}$ is the time until which the final $\id$ generated from $S^*$ was broadcast. The server checks $t_{start}$ and $t_{end}$ are reasonable before accepting them. The user erases its memory, and generates a new initial seed $S_0$. 
%\TODO{Y: Who is allowed to upload to the server?  %Do you need a doctor's permission? }
% Clarified below.
\item {\bf Checking proximity.} A user wanting to check proximity to a positive individual downloads the list $\mathcal{L}$ from the server. For every seed $(S^*,t_{start})$ in $\mathcal{L}$, it generates the sequence of IDs $\id^*_1, \ldots, \id^*_{\Delta}$ starting from $S^*$, as well as times $t_i = t_{start} + (i-1) \dt$. If $\mathcal{S}$ contains $(\id_i^*, t)$ for some $i \in \{1, \ldots, \Delta\}$ such that $t$ and $t_i$ are sufficiently close, the user is alerted of potential exposure.
\end{itemize}
{\bf Setting delays.} To prevent replay attacks, we need to publish the data with a slight delay. The delay should be chosen such that the last $\id^*_{\Delta}$ generated from $S^*$ is not recognized as potential exposure by any user if immediately rebroadcast by a malicious party. We envision this delay to be a few minutes (e.g., 15 minutes).

%{\bf Messages.} Optionally, entry can include a message from the user, to be released to 

\noindent {\bf Entry validation.} Entries on the server can be validated by attaching a variable-length sequence of signatures on them. To this end, the server offers an interface SignEntry(entryPtr, [data], signature, [cert]) which allows an entity to attach a signature on entry (described by a pointer entrPtr), and some (optional) metadata. A certificate to verify the verification key of the signing entity can also be attached.

This approach allows for different policies, which we do not specify the policies further here but note that a range of designs are supported by the approach. Several certifying entities (typically, healthcare providers, proxy organizations, etc.) can adopt different procedures to append their signatures. Upon an initial update, a (weakly secure) signature with an app-specific key could be attached. Similarly, users can adopt different policies when an exposure is reported depending on the attached signatures. We also do not specify here the infrastructure required to establish the validity of certificates, or how a user interacts with a signing party to prove ownership of an entry, as this depends on processes and policies outside the scope of this description of core privacy protocol.


\noindent {\bf Privacy/Security properties.} Let us refer to users as either ``positive'' or ``negative'' depending on whether they decided to report as positive or not. All of the following statements assume cryptographic security of $f$. Then, the above protocol has the following properties:
\begin{itemize}
    \item {\em Privacy for negative users.}  A negative user $u$ only broadcasts pseudorandom IDs. These IDs cannot be linked without knowledge of the internal state of $u$. This privacy guarantees improves with the frequency of updating the seed $S_i$ -- ideally, if a different $\id_i$ is broadcast each time, no linking is possible. This however results in less efficient checking for exposure by negative users.
    \item {\em Privacy for positive users.} Upon reporting positive, the last $\Delta$ IDs generated by the positive user {\em can} be linked. Note that this is only true for the IDs generated within the infection window. Older IDs and {\em newer} IDs cannot be linked with those in the infection window, and with each other.
    \item {\em Integrity guarantees.} It is infeasible for an attacker to upload to the server a value $S^*$ which generates an $\id$ that equals one generated by another user. This prevents the attacker from misreporting negative users, and erroneously alert their contacts.  
    \item {\em Resilience to replay attacks.} The time stamping is necessary to prevent replay attacks -- we want to prevent an attacker to simply download an $\id$ from the server, and broadcast it in a wide region, erroneously convincing a large number of users of potential exposure.
\end{itemize}

\noindent {\bf Extensions.} One concern is that IDs of a positive user can be linked within the infection window. This concern can be mitigated. Instead of maintaining a single sequence $S_0, S_1, \ldots$, a user can regularly restart with a random initial seed, and generate  shorter sequences. Upon reporting positive, the user uploads all seeds generating sequences used within the infection window. The server however should not simply append uploaded seeds -- rather, it should collect seeds for a sufficiently long period, and shuffle them. The actual privacy improvement is  hard to assess without a good statistical model of upload frequency. This also increases the latency of the system.

\subsection{Mobile-assisted Contact Tracing Interviews} 
Contact tracing interviews are laborious and often miss important events due to the limitations of human memory.  Our plan to assist here is to provide information to the end user (and with consent) that can be shared with a healthcare organization which is charged with performing contact tracing interviews.   This is not an exposure of the entire observational log, but rather an extract of the information which is requested in a standard contact tracing interview. We have been working with healthcare teams from Boston and the University of Washington on formats and content of information that are traditionally sought by public health agencies.

\subsection{Narrowcasting}

Healthcare authorities from NYC have informed us that they would love to have the ability to make public service announcements which are highly tailored to a location or to a subset of people who may have been in a certain region during specific periods of time.  This capability can be enabled with a public server supporting (area x time,message) pairs.  Here "area" is a location, a radius (minimum 10 meters), a beginning time and an ending time.  Only announcements from recognized public health authorities are allowed.  

Anyone can manually query the public server to determine if there are messages potentially relevant to them per their locations and dwells at the locations over a period of time. However, simple automation can be extremely helpful as phones can listen in and alert based on filters that are dynamically set up based on privately-held locations and activities. Upon downloading (area x time, message) pairs a phone app (for example) can automatically check whether the message is relevant to the user.  If it is relevant, a message is relayed to the device owner.

Querying the public server provides \emph{no} information to the server through the protocol itself, because only a simple copy is required.  

\subsection{Alternative: Absolute-Location--Centric Mobile Tracing Methods}\label{sect:GPS}
It is also possible to design protocols based on the sensing of absolute locations (GPS, and GPS extended with dead reckoning, wifi, other signals) consistent with "If you do not report as being positive, then no information of yours will leave your phone." (see Section~\ref{sect:FAQ}).  For example, a system could upload location traces of positives (crytographically, in a secure manner), and then negative users, whose traces are stored on their phones could intersect their traces with the positive traces to check for exposure. This method could potentially be done with stronger cryptographic protocols to be more privacy sensitive to the positive user.

There are two reasons why we do not focus on the details of the such an approach here, the first of which is more severe:
\begin{itemize}
    \item Current localization technologies are not as accurate as the use of Bluetooth-based proximity detection, and may not be accurate enough to be consistent with medically suggested definitions for exposure.
    \item Approaches employing the sensing and collection of absolute location information would need to rely more heavily on cryptographic protocols to keep the positive users traces secure.
\end{itemize}
However, this is a protocol worth keeping in mind as an alternative approach, per assessments of achievable accuracies.

\section{Further Considerations and Concerns}
We now enumerate a list of further privacy and security considerations.

\subsection{Ethics Considerations}
We acknowledge that ethical questions arise with contact tracing and in the development and adoption of any new technology. The question of how to balance what is revealed for the good of public health vs individual freedoms is one that is central to public health law. In many nations, positively tested citizens are required to disclose aspects of their history.  The purpose of this document is lay out some of the technological capabilities, which supports discussion and debate.

%This technology enables more effecting public health law in manner that is privacy sensitive.
\iffalse
\begin{enumerate}
\item All tracing technologies come with risks, benefits, and tradeoffs, including inescapable false positive and false negative rates. The false alerts may not occur evenly but by particular demographic attributes, putting a unfair burdens on specific demographic groups. Accordingly, alerts themselves may also benefit one cross section of society more than another. It is worth noting the current COVID-19 pandemic has placed highly imbalanced risks on essential workers (particularly healthcare workers) compared to non-essential workers in developed nations.
\item The narrowcast broadcasting mechanism permits the local public health departments to provide messages to citizens. The quality of local health departments may vary widely in different regions thereby enabling further imbalances.
\item If adoption occurs, then this creates a different equilibrium in society, where citizens have adopted a new technology which will impact their lives and health. It may not benefit all parties equally.
\item This technology may provide the greatest benefit to those with mobile phones.
\item The protocols outlined here do not explicitly require government regulation or hospital certification. They may also be used with crowd-sourcing or other distributed means. This has implications to lesser developed nations.
\end{enumerate}
\fi

%A detailed list of all potential ethical concerns is beyond the scope of this document, at the current time. 

\subsection{Larger Considerations of Testing, Tracing, and Timeouts}
Tracing is part one part of a conventional epidemic response strategy, based on Tests, Tracing, and Timeouts (TTT). Programs involving all three components are as follows:
\begin{itemize}
\item Test heavily for the virus.  South Korea ran over 20 tests per person found with the virus. 
\item Trace the recent physical contacts for anyone who tests positive.  South Korea conducted \emph{mobile contact tracing} using telecom information.
\item Timeout the virus by quarantining contacts until their immune system purges the virus, rendering them non-infectious.
\end{itemize}
The mobile tracing approach allows this strategy to be applied at a dramatically larger scale than only relying on human contact tracers.  

\subsection{Challenge of Wide-scale Adoption}
This chain is only as strong as its weakest link.  Widespread testing is required and wide-scale adoption must occur. Furthermore, strategies must also be employed so that citizens takes steps to self-quarantine or take further tests when they are exposed. We cannot assume 100 percent usage of the application and concomitant enlistment in TTT programs.  Studies are needed of the efficacy of the sensitivity of the effectiveness of the approach to to different levels of subscription in a population.

\appendix

\section{Issues around Practical Implementation}

A number of practical issues and details may arise with implementation.
\begin{enumerate}
    \item With regards to anonymity, if the protocol is implemented over the internet, then geoip lookups can be used to localize the query-maker to a varying extent.  People who really care about this could potentially query through an anonymization service.
    \item The narrowcast messages in particular may be best expressed through existing software map technology.   For example, we could imagine a map querying the server on behalf of users and displaying public health messages on the map.  
    \item The bandwidth and compute usage of a phone querying the full database may be to high.  To avoid this, it's reasonably easy to augment the protocol to allow users to query within a (still large) region.  We mention one such approach below.  
    \item Disjoint authorities.  Across the world, there may be many testing authorities which do not agree on a common infrastructure but which do wan to use the protocol.  This can be accommodated by enabling the phone app to connect to multiple servers. 
\end{enumerate}

There are several ways to implement the server.  A simple approach, which works fine for not-to-many messages just uses a public github repository.

A more complex approach supporting regional queries is defined next.  

\subsection{regional query support}
Anyone can ask for a set of messages relevant to some region $R$ where $R$ is defined by a latitude/longitude range with messages after some timestamp.  More specific subscriptions can be constructed on the fly based on policies that consider a region $R$ and privately observed periods of time that an individual has spent in a region. Such scoped queries and messaging services that relay content based on location or on location and periods of time are a convenience to make computation and communication tractable.  The reference implementation uses regions greater in size than typical GeoIP tables.

To be specific, let's first define some concepts.
\begin{itemize}
    \item Region: A region consists of a latitude prefix, a longitude prefix, and the precision in each.  For example, New York which is at 40.71455 N, -74.00712 E can be coarsened to 40 N, -74 E with two digits of precision (the actual implementation would use bits).
    \item Time: A timestamp is specified in the number of seconds (as a 64 bit integer) since the January 1, 1970. 
    \item Location: A location consists of a full precision Latitude and Longitude
    \item Area: An area consists of a Location, a Radius, a beginning Time, and an ending Time.
    \item Bluetooth Message: A Bluetooth message consists of a fixed-length string of bytes.  It is used with the Bluetooth sensory log to discover if there is a match, which results in a warning that the user may have been in contact with an infected person.  
    \item Message: A message is a cryptographically signed string of bytes which is interpreted by the phone app. This is used for either a public health message (announced to the user if the sensory log matches) or a Bluetooth Message. 
\end{itemize}

With the above defined, there are two common queries that the server supports as well as an announcement mechanism.
\begin{itemize}
    \item GetMessages(Region, Time) returns all of the (Area, Message) pairs that the server has added since Time for the Region. The app can then check locally whether the Area intersects with the recorded sensory log of (Location,Time) pairs on the phone, and alert the user with the Message if so.   
    \item HowBig(Region, Time) returns the (approximate) number of bytes worth of messages that would be downloaded on a GetMessages call with the same arguments.  HowBig allows the phone app to control how much information it reveals to the server about locations/times of interest according to a bandwidth/privacy tradeoff.  For example, the phone could start with a very coarse region, specifying higher precision regions until the bandwidth required is acceptable, then invoke GetMessages.  (This functionality is designed to support controlled anonymity across widely varying population densities.)
    \item Announce(Area,Message) uploads an (Area, Message) pair for general distribution.  To prevent spamming, the signature of the message is checked against a whitelist defined with the server. 
\end{itemize}

\end{document}

