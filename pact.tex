\documentclass{article}
\usepackage[english]{babel}
\usepackage[utf8]{inputenc}
\usepackage{hyperref}
%\title{{\Huge PACT\/}: Open Source {\Huge P\/}rotocols %{\Huge A\/}nd Standards \\
%for Mobile {\Huge C\/}ontact {\Huge T\/}racing %Applications}
%\title{Privacy-Sensitive Mobile Contact Tracing: Protocols and Standards}
\title{{\Huge PACT\/}:   {\Huge P\/}rivacy Sensitive Protocols {\Huge A\/}nd Standards\\
for Mobile {\Huge C\/}ontact {\Huge T\/}racing }
\date{}

\setlength{\oddsidemargin}{0.25 in}
\setlength{\evensidemargin}{-0.25 in}
\setlength{\topmargin}{-0.6 in}
\setlength{\textwidth}{6.5 in}
\setlength{\textheight}{8.5 in}
\setlength{\headsep}{0.75 in}
\setlength{\parindent}{0 in}
\setlength{\parskip}{0.1 in}


\begin{document}
\maketitle

As nations are seeking to circumvent devastating death tolls from COVID-19, many are resorting to \emph{mobile-based, contact tracing technologies}. Harnessing geo-sensing mobile computing technologies
is an obvious means to dramatically scale up conventional epidemic response strategies
to do tracking at population scale. However, straightforward and well-intentioned contact tracing applications can invade personal privacy and provide justification for data collection and mass surveillance that are inconsistent with the civil liberties that citizens will and should expect--and demand. To be effective, acceptable, and consistent with expectations for privacy, we must leverage designs and computing advances in privacy and security. And in cases where it is valuable for individuals to share data with others, systems must provide voluntary mechanisms in accordance with ethical principles of personal decision making, disclosure, and consent. We focus here on privacy-sensitive mobile tracing technologies that are designed to mitigate COVID-19, while upholding civil liberties.  We refer to efforts to identify, study, and field  privacy-sensitive technologies, architectures, and protocols in support of mobile tracing as PACT (\emph{P}rivacy sensitive protocols \emph{A}nd standards for mobile \emph{C}ontact \emph{T}racing).

\begin{center}
\emph{The objective of PACT is to set forth transparent privacy and
  anonymity standards,\\
  which permit adoption of mobile contract tracing efforts while upholding civil liberties.}
\end{center}

We seek to specify a set of protocols and standards that achieve
these objectives. We also wish to provide a means to support the concurrent use of multiple, mobile contact tracing applications via providing interoperating capabilities that assure consistency with regards to the goals of privacy.

Conventional contact tracing strategies executed by public health organizations operate as follows: Positively tested citizens must (by law) reveal their contact history to public health officers. The public health officers then inform other citizens who have been exposed to the virus based on co-location, via a definition of co-location, supported by look-up or inference about locations. The citizens deemed to have been exposed are then asked to take appropriate action (often to either seek tests or to quarantine
themselves and be vigilant about symptoms).  

A mobile contact-tracing application seeks to augment the services provided public health offices, by enabling the following capabilities via computing and communications technology:

\begin{itemize}
\item \textbf{Efficient interviews:}  If a citizen becomes ill, he or she can use the app to improve the efficiency and completeness of manual contact tracing interviews. Privacy here is maximized by ensuring that all the data remains on the user's device, except for what they voluntarily reveal to healthcare authorities in order to enable contact tracing.

\item \textbf{Narrowcast messages:}  Public health authorities can broadcast
  custom-tailored messages to specific, relevant subsets of citizens.  For example, the following message might be issued: ``If you visited
  the X Eldercare Center between March 7th and 10th, please email yy@hhhealth.org''.  Here, it is a citizens voluntary choice to search these messages for what is relevant to them.
  
\item \textbf{Privacy-sensitive, distributed alerting:}  The app can enable someone you were near in the last two weeks to warn you that they are now ill, even if you do not know them personally.  Crucially, as with conventional contract tracing done by public health services, the alert mechanism has a \emph{provably} secure implementation so as to ensure that positives cases remain anonymous to the general public, and that the alerts to exposed citizens are informed in a provably secure manner.
\end{itemize}

Importantly, the smartphone app, by default, keeps \emph{all} personal data on a citizen's phone, while enabling these key capabilities; information is shared via voluntary disclosure actions taken, with the understandings relayed via careful disclosure. 

\emph{From a civil liberties standpoint, the privacy guarantees these protocols ensure are consistent with the disclosures already extant in contract tracing methods done by public health services} (where some information from a positive tested citizen is revealed to other exposed citizens). In short, we seek to empower public health services, while maintaining civil liberties.
 
\section{Protocols and Standards} 
We now overview these three functionalities.

\subsection{Contact Tracing Interviews} 
Contact tracing interviews are laborious and often miss important events due to the limitations of human memory.  Our plan to assist here is to provide information to the end user (and with consent) that can be shared with a healthcare organization which is charged with performing contact tracing interviews.   This is not an exposure of the entire observational log, but rather an extract of the information which is requested in a standard contact tracing interview. We have been working with healthcare teams from Boston and the University of Washington on formats and content of information that are traditionally sought by public health agencies.

\subsection{Narrowcasting}

Healthcare authorities from NYC have informed us that they would love to have the ability to make public service announcements which are highly tailored to a location or to a subset of people who may have been in a certain region during specific periods of time.  This capability can be enabled with a public server supporting (area x time,message) pairs.  Here "area" is a location, a radius (minimum 10 meters), a beginning time and an ending time.  Only announcements from recognized public health authorities are allowed.  

Anyone can manually query the public server to determine if there are messages potentially relevant to them per their locations and dwells at the locations over a period of time. However, simple automation can be extremely helpful as phones can listen in and alert based on filters that are dynamically set up based on privately-held locations and activities. Upon downloading (area x time, message) pairs a phone app (for example) can automatically check whether the message is relevant to the user.  If it is relevant, a message is relayed to the device owner.

The public server supports scoped queries.  In particular, anyone can ask for a set of messages relevant to some region $R$ where $R$ is defined by a latitude/longitude range with messages after some timestamp.  More specific subscriptions can be constructed on the fly based policies that consider region $R$ and privately observed periods of time that an individual has spent in a region. Such scoped queries and messaging services that relay content based on location or on location and periods of time are a convenience to make computation and communication tractable.  The reference implementation uses regions greater in size than typical GeoIP tables.

\subsection{Privacy-sensitive, distributed alerting.} 
One approach to privacy-sensitive alerting is to perform distributed, community-sourced approach to sharing information. In an example implementation, proximity-detecting signals via Bluetooth or ultrasonics can be harnessed.  If a user is both infected and is willing to warn others who may have been at risk via proximity to the user, then an authorization is given by healthcare authorities so as to warn people at risk.

An example protocol with Bluetooth signals is as follows:

Let $u$ be the information that the Bluetooth device announces to the world.

The phone periodically replaces $u$ with a random number.

The phone scans for Bluetooth devices every minute.  For any bluetooth device it finds with id $d$,the phone stores $V = \mbox{hash}(d,u)$ where hash is a commonly known cryptographically strong hash.

If someone wants to pass a message m to contacts, they upload $R,(V,m)$ for each contact in their log.  Here $R$ is a region as defined for narrowcasting.

The phone periodically scans a public database, downloading tokens $(V_1,m_1), (V_2,m_2),..., (V_n, m_n)$ associated with people who are positive.  For each entry $V$ in the phone’s storage and for each $V_i$, the phone checks whether $V_i=V$ and if so, announce $m$ to the user.

The same public server can support these queries. 

\end{document}

