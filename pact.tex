\documentclass{article}
\usepackage[english]{babel}
\usepackage[utf8]{inputenc}
\usepackage{hyperref}
\title{{\huge PACT\/}: Open Source {\Huge P\/}rotocols {\Huge A\/}nd Standards \\
for Mobile {\Huge C\/}ontact {\Huge T\/}racing Applications}
%\title{CovidSafe: Private Test, Trace, and Timeout}
\date{}

\begin{document}
\maketitle

Running unchecked, the death toll of COVID-19 is estimated to be in the 10s of millions or more.  To avoid this, economy killing lockdowns are being imposed as shown in Hubei.  South Korea however managed to deal with an 8000-person outbreak initiated with a super-spreading event (\href{https://www.propublica.org/article/how-south-korea-scaled-coronavirus-testing-while-the-us-fell-dangerously-behind}{Propublica} \href{https://www.sciencemag.org/news/2020/03/coronavirus-cases-have-dropped-sharply-south-korea-whats-secret-its-success}{Science Magazine}) showing that a third strategy consisting of a three link chain works:
\begin{itemize}
\item Test heavily for the virus.  South Korea ran over 20 tests per person found with the virus. 
\item Trace down the contacts for anyone who tests positive.  South Korea conducted \emph{mobile contact tracing} using telecom information.
\item Timeout the virus by quarantining contacts until their immune system purges any virus.
\end{itemize}
The TTT approach is a conventional epidemic response strategy applied
at dramatically more scale than before.  Each link in the chain is
required for success so the use of telecom information is a genuinely
worrisome precedent.  What if a government decides the opposition
party is a public health emergency?

Private Test Trace Timeout (PTTT) methods can empower the public and
health authorities to respond to COVID-19 while avoiding tools
reusable for oppression.  The key element here is a smartphone app
which keeps \emph{all} personal data on your phone by default while
enabling key capabilities:
\begin{itemize}
\item Crowd-sourced warnings.  The app can enable someone you were near in the last two weeks to warn you that they are ill, even if you do not know them personally.  This ability is cryptographically secure, retroactively usable, and unspammable. 
\item Efficient interviews.  If you become ill, you can use the app to improve the efficiency and completeness of manual contact tracing interviews.
\item Narrowcast messages.  Public health authorities can broadcast messages with very narrow relevance.  For example ``If you visited this nursing home 1-4 days ago, please email x@y.z.''
\end{itemize}
Even in areas with a massive outbreak, a weeks-long lockdown can bring the number of new cases per day under control so that CovidSafe PTTT can control the virus until a vaccine is developed.

\section{Privacy Guarantees}

\section{Implementation}

\end{document}





It is possible to keep people alive and have an economy in a democracy. 

Running unchecked, the death toll of COVID-19 is estimated to be in
the millions in the USA1. South Korea provides the best evidence that
this is avoidable using mobile contact tracing, while maintaining a
functioning economy;  they dealt with an 8000-person outbreak
generated by a super-spreading event in a church while avoiding severe
economically-damaging lockdowns (Propublica Link \& Science Mag Link
).  Additional evidence is provided by China (outside of Hubei),
Taiwan, and Singapore, all of which have effectively dealt with
outbreaks without shutting down their economy.  In addition to South
Korea’s mobile contact tracing effort2, Singapore has recently
released a mobile contact tracing app, TraceTogether (Link); Israel
passed an emergency law permitting the use of mobile data for COVID-19
contact tracing (Link); the United Kingdom looks also to be developing
a mobile contact tracing app (Link, Link, and Paper).    

 
COVID-19 can be addressed by allowing people to become infected,
resulting in millions of deaths.  Alternatively, the pandemic can be
addressed by locking down everyone in quarantines, resulting in a
massive depression, which in itself can be linked to morbidity and
deaths (cite: there are some economic studies on this).  It can also
be addressed by enabling more relaxed controls that would better
sustain the economy and the harnessing of contact tracing.   To date,
contact tracing has been linked to government services and
surveillance.  We believe that we must be extremely cautious about
governments arguing that COVID-19 justifies state-based surveillance.
There is a risk that authoritarian regimes engage in further
incursions into personal privacy in the name public health and that
democracies that have long protected the basic human right of privacy
would weaken this commitment in light of the pandemic.  We believe
that the creative application of computing technologies and
architectures can provide a privacy-preserving methodology for
testing, tracing, and timeout (TTT).  We refer to the family of
privacy-preserving TTT as PTTT methods.  Here we describe a specific
approach to PTTT that we refer to as CovidSafe.  

 
CovidSafe enhances and automates existing manual contact tracing in a
manner which preserves privacy and personal autonomy.   Even in areas
with a massive outbreak, we expect that weeks-long lockdown can be
used to bring the number of new cases per day under control, then
CovidSafe can empower TTT to control the virus while keeping the
economy going until a vaccine and/or more effective critical care
medicine is developed.    


We are currently coordinating with the University of Washington to
ensure access to validated positive tests; the University of
Washington is conducting 90\% of COVID19 tests in Washington State
(Link).   

 

 

To succeed, we need to do three things in a way that is both rapid and scalable: Test aggressively for the virus; Trace the contacts (e.g., with a threshold such as >1% chance of infection); and Timeout these traced contacts through evidence-based public health interventions based on level of exposure (e.g., immediate self-quarantine vs. serial symptom check, Link).   

 

The Test/Trace/Timeout (TTT) approach is only as strong as the weakest link.  Our goal is to support the contact tracing component of this strategy through empowering users to know whether they have been exposed and what to do in a privacy preserving manner that supports population health management at a national scale.  We envision TTT as the ‘backbone’ of a combined response strategy as suggested by the WHO (Link).    

\subsection*{Privacy and Voluntary Disclosure Protocols and Implications}

The system is designed to maintain strong privacy and anonymity
protections to the user, while utilizing voluntarily disclosed
information for societal benefits.  The moral, mathematical, and
engineering responsibilities of the system are to respect a set of
well specified privacy preserving principles. Importantly, it should
be emphasized that user data is held by the user, until the user
voluntarily discloses information. The below outlines the anonymity
guarantees, which the system is designed to respect.

\textbf{Voluntarily and secure disclosure to help public health
  services.}  Currently, contact tracers in public health
organizations interview a positively confirmed person in order to
determine other exposures in the population, along with accessing
other high risk situations (e.g. visits to nursing care centers). The
contract tracer then informs those relevant individuals in the
population of potential risks. The prototype system securely maintains
location information (maintained on the users devices), which the user
may voluntarily disclose only to a public health official, which the
public health official may use in the interview process.

\textbf{Anonymized disclosure to inform community members of risks.}
A positively tested user is able to anonymously inform the community
of any possible exposure. In particular, any participating community
member will be anonymously informed if they are potentially at risk
due to being in an exposure event, which has occurred in some
pre-specified recent time period. All community members will not be
explicitly informed of anything more than this. Importantly, from a
precise security perspective, no further private data (including
identities) will be revealed to public. This is consistent with what
is provided by public health contact tracing services.

\textbf{Regional support for public health announcements. } The app
will support a means of mobile communication between public health
officials and users in a crude geographic region, e.g. so that public
health officials can broadcast helpful information to users in some
pre-specified region (e.g. a town). The user can optionally decide to
share their crude location (e.g. the town they are in) so that they
can receive up-to-date information from public health services.
